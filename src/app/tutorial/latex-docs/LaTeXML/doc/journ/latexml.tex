\section{The {\latexml} {\LaTeX} to {\xml} Converter}\label{sec:latexml}

\begin{oldpart}{@BRUCE, This is an old overview, this is where your part comes in.}
The {\latexml} system, consists of a {\TeX} parser, an {\xml} emitter, and a
post-processing pipeline. To cope with {\LaTeX} documents, the system needs to
supply {\twindef{\latexml}{bindings}} (i.e. special directives for the {\xml}
emitter) for the {\LaTeX} packages. So every {\LaTeX} package {\tt{package.sty}}
comes with a {\latexml} binding file {\tt{package.ltxml}}, a {\perl} file which
contains {\latexml} constructor-, abbreviation-, and environment definitions, e.g.

\begin{footnotesize}
\begin{verbatim}
DefConstructor("\Reals","<XMTok name='Reals'/>");
DefConstructor("\SmoothFunctionsOn{}","<XMApp><XMTok name='SmoothFunctionsOn'/>#1</XMApp>");
DefMacro("\SmoothFunctionsOnReals","\SmoothFunctionsOn\Reals");
\end{verbatim}
\end{footnotesize}  

{\tt{DefConstructor}} is used for semantic macros, whereas {\tt{DefMacro}} is used
for abbreviative macros. The latter is used, since the {\tt{latexml}} program does
not read {\tt{package.sty}} and needs to be told, which sequence of tokens to
recurse on. The {\latexml} distribution contains {\latexml} bindings for the most
common base {\LaTeX} packages.

For the {\xml} conversion, the {\tt{latexml}} program is run, say on a file
{\tt{doc.tex}}. {\tt{latexml}} loads the {\latexml} bindings for the {\LaTeX}
packages used in {\tt{doc.tex}} and generates an {\xml} file {\tt{doc.tex.xml}},
which closely mimics the structure of the parse tree of the {\LaTeX} source. The
structure of this file is determined by the {\latexml} packages and must
correspond to a format-specific document type definition (DTD), which is usually a
relatively simple extension of Bruce Miller's {\LaTeX} DTD.

In the semantic post-processing phase, the {\LaTeX}-near representation in
{\tt{doc.tex.xml}} is transformed into the target format by the {\tt{latexmlpost}}
program. This program applies a pipeline of intelligent filters to
{\tt{doc.tex.xml}}. The {\latexml} program supplies various filters, e.g. for
processing {\html} tables, including graphics, or converting formulae to
{\pmathml}. Other filters like transformation to {\openmath} and {\cmathml} are
currently under development. The filters can also consist of regular
{\xml}-to-{\xml} transformation process, e.g.  an {\xslt} style sheet. Eventually,
post-processing will include semantic disambiguation information like types,
part-of-speech analysis, etc. to alleviate the semantic markup density for
authors.
\end{oldpart}

%%% Local Variables: 
%%% mode: stex
%%% TeX-master: "main"
%%% End: 
